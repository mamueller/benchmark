\documentclass[twoside,10pt]{article}    
\usepackage{graphicx,makeidx,multicol,footmisc} % Packages for svmult %   
%\usepackage{mathptmx,helvet,courier}        % selects fonts
\usepackage[numbers,sort&compress]{natbib}
%\usepackage{cite}
\usepackage{textcomp,color,german}
\usepackage[rightcaption,ragged]{sidecap}
%\usepackage[left=2cm,right=2cm,top=1.5cm,bottom=1.0cm,nohead,nofoot]{geometry}
\setlength{\paperheight}{29.7cm}
\setlength{\paperwidth}{21.0cm}
\setlength{\topmargin}{-1.04cm}
\setlength{\headheight}{0.0cm}
\setlength{\headsep}{0.0cm}
%\setlength{\topskip}{0.0cm}
\setlength{\textheight}{26.3cm}
\setlength{\footskip}{0.8cm}
\setlength{\textwidth}{16.7cm}
\setlength{\oddsidemargin}{-0.04cm}
\setlength{\evensidemargin}{-0.74cm}
\setlength{\bibsep}{1mm}
\newcommand{\pdiff}[2]{\frac{\partial#1}{\partial #2}}
\newenvironment{dlist}
   {\begin{list}
      {$\bullet$}
      {
      \setlength{\topsep}{0.5ex}
      \setlength{\partopsep}{0.0ex}
      \setlength{\parsep}{0.5ex}
      \setlength{\itemsep}{0.0ex}
      \setlength{\itemindent}{3.0ex}
      \setlength{\leftmargin}{0.0ex}
      \setlength{\labelsep}{1.0ex}
      }
   }
   {\end{list}}

\graphicspath{{bilder/}}

\selectlanguage{USenglish}
\frenchspacing
%\makeindex

%\hyphenation{EkinUM}

\begin{document}
\sffamily
%\title{Dynamic Modeling of the Andean Orogeny Embedded in a 3D Spherical-Shell 
%Model of the Earth's Mantle}
\title{Application for Renewal of a Research Grant - 
  Fortsetzungsantrag auf Sachbeihilfe}
\date{}
%\titlerunning{}
%\author{Uwe Walzer\inst{1}\and Roland Hendel\inst{1}}
%\maketitle


%% ------------------------------------------------------------------------ %%
%
%  SECTION 1
%
%% ------------------------------------------------------------------------ %%

\section{General Information (Allgemeine Angaben)}
This is a renewal proposal for a research grant.
%\\[4ex]
%{\LARGE Dynamic Modeling of the Andean Orogeny Embedded in a 3D Spherical-Shell 
%Model of the Earth's Mantle}

\subsection{Applicants (Antragsteller)}
\noindent
\hspace*{-2.83mm}
\begin{tabular}{p{8.2cm}p{8.2cm}}
\textbf{Project leader} & \textbf{Second applicant} \\
Jonas Kley, Prof. Dr.& Uwe Walzer, Prof. Dr.\\
Professor for Structural Geology and Tectonics & Professor for Geophysics \\
Born 06.04.1961, German & Born 31.05.1941, German \\
Last DFG-Project: KL 495/15-1 & Last DFG project: WA 1035/5-3 \\
Institut f\"ur Geowissenschaften, Friedrich-Schiller-Univ. &
Institut f\"ur Geowissenschaften, Friedrich-Schiller-Univ.\\
Burgweg 11, 07749 Jena, Tel. 03641-948623 & Humboldtstr. 11, 07749 Jena, Tel. 03641-948683 \\
Fax 03641-948622, E-mail: jonas.kley@uni-jena.de & Fax 03641-948662, E-mail: u.walzer@uni-jena.de \\
Home: Haydnstr. 13, 07749 Jena, Tel. 03641-478797 & Home: In der Hohle 5, 07751 Jena, Tel. 03641-827751 \\
 & \\
\textbf{Third applicant} \\
Lothar Viereck-G\"otte, Prof. Dr.\\
Professor for Geochemistry \\
Born 27.05.1952, German \\
Last DFG-Project: VI 215/14-2\\
Institut f\"ur Geowissenschaften, Friedrich-Schiller-Univ.\\
Burgweg 11, 07749 Jena, Tel. 03641-948720 \\
Fax 03641-948662, lothar.viereck-goette@uni-jena.de \\
Home: L\"utzerodaer Str. 6, 07751 Jena, 036425-50931 &
\raisebox{10.77ex}[-10.77ex]{
\hspace*{-2.3mm}
\parbox{8.2cm}{
Regarding computational physics the project is in cooperation with the five 
persons denoted by an asterisk in Section \ref{sec:coop}, and especially with 
John R. Baumgardner, Ph.D. in Geophysics/ Space Physics, Born 23.01.1944, USA. 
Univ. of California, Dep. %artment 
of Earth and Planet. Science, 307 McCone Hall, Berkeley, CA, 94720-4767, USA.
Tel. 001 760 440-9345, E-mail: jrbaumgardner@cox.net
Home: 24515 Novato Place, Ramona, CA 92065, USA.
Dr. Baumgardner does \emph{not} ask for DFG funding.}
} 
\\
\end{tabular}\\

\subsection{Topic (Thema)}
Dynamic Modeling of the Andean Orogeny Embedded in a 3D Spherical-Shell 
Model of the Earth's Mantle. \\
%(Dynamische Modellierung der Andenorogenese, eingebettet in ein 3D Kugelschalenmodell des Erdmantels), \\
South America

%\subsection{Code name (reference) (Kennwort)}
%Model of Andean Orogeny

\subsection{Research area and field of work (Fach- und Arbeitsrichtung)}
Structural geology, geophysics, geodynamics, numerical modeling, mineralogy

\subsection{Anticipated total duration (Voraussichtliche Gesamtdauer)}
This is a renewal proposal. The duration is three years, yet: October~1, 2012 -- September~30, 2015

\subsection{Application period (Antragszeitraum)}
Here we ask for a funding period of 36 months. The previous grant was given on Dec.~12, 2008.
Current funding for personnel lasts until Jan.~31, and Sept.~30, 2012, for R.~Hendel and C.~K\"ostler, respectively.
Funding for direct project costs lasts until Sept.~30, 2012.
We wish funding of the renewed project to begin at Oct.~1, 2012, 
but we ask for a bridge financing for R.~Hendel from Feb.~1, 2012 to Sept.~30, 2012.

\subsection{Summary (German text, see Appendix)}
Some essential features of Andean orogenesis cannot be explained \emph{only} by a dynamic \emph{regional} model since there are essential influences across its vertical boundaries.
A dynamic regional model of the Andes should be embedded in a 3-D spherical-shell model. 
Because of the energy distribution on the poloidal and toroidal parts of the creeping velocity and because of geologically determined mass transport alongside the Andes, both models have to be 3-D. 
We developed a new viscosity profile of the mantle with jumps at the lithospheric-asthenospheric boundary and at a depth of 410, 520 and 660 km.
Therefore, the challenges to the code Terra are now essentially larger.
In the last three years we have resolved these problems in an international cooperation (see \ref{sec:prelim}~b)).
Based on the new viscosity profile and on the improved Terra, we computed a new forward spherical-shell model \cite{Walzer2012,Walzer2012b}. 
Papers on the improvement of Terra \cite{Koestler2012,Mueller2012} have been submitted. 
We conceived a regional model of the Andean orogenesis (\ref{sec:sched}~b)) with the same new viscosity profile. 
We want to investigate why there is flat-slab subduction in some segments of the Andes and why deformation of the crust and volcanism migrate eastward. 
The evolution of the abundances of incompatible elements indicate a cycle which was finished by a fast process, perhaps by a large-scale delamination of the lower plate, perhaps also by another type of delamination. 
In connection with another spherical-shell model (with prescribed plate boundaries), the regional model should numerically explain why a plateau-type orogen evolved at an oceanic-continental plate boundary.

%% ------------------------------------------------------------------------ %%
%
%  SECTION 2
%
%% ------------------------------------------------------------------------ %%

\section{State of the art, preliminary work (Stand der Forschung, eigene Vorarbeiten}
\subsection{State of the art (Stand der Forschung)}
The papers which refer to our topic can be classified by six subjects.\\
a) geological description of dynamic problems of the Andean orogeny,\\
b) models which are partially kinematic and partially dynamic. 
In this kind of models, essential features are prescribed in order to gain a large adaptation to geological and geophysical observations.\\
c) geochemical models of growth and differentiation of continents which do not contain any dynamic modeling,\\
d) dynamic models of the subduction process to understand the physical mechanism behind subduction, \\
e) circulation models, \\
f) fully dynamic models of subduction in a diamond (cf. Fig. \ref{fig:4}), i.e. in a certain 3-D sector of the spherical shell, which represents the mantle, where this diamond is embedded into a realistic 3-D spherical-shell solution.

We systematically described the papers of types a) to e) by other authors in our \emph{initial} proposal, three years ago. 
Therefore, we will not repeat it here.
Up to now, there is no paper of type f). 
Some supplements will follow: To obtain a more profound analysis of the Andean orogeny, it is important to understand why a plateau-type orogen formed between a purely oceanic lithospheric plate (Nazca plate) and a continent (South America) \cite{Oncken2006}. 
The Andean mountain belt belongs to the non-collisional type. 
Kley and Monaldi \cite{Kley1998,Kley1999,Kley2002} found a Cenozoic shortening of 250-350 km whereas Arriagada et.~al. \cite{Arriagada2008} derived 400~km for the central Andes. 
In other areas of the Earth, however, the subduction zones at an oceanic-continental plate-boundary site have only little or no shortening and do not show any elevated plateaus. 
In most cases, the upper plate is characterized by backarc extension. 
Schellart and Rawlinson \cite{Schellart2010} discuss some hypotheses of which it is claimed that they explain this exceptional behavior of the Andes.
\begin{dlist}
 \item In \cite{Molnar1978}, it is proposed that the young age of the Nazca plate and low negative buoyancy cause this phenomenon.
 \item Climatic conditions, high friction and subduction erosion is thought to be the decisive factor \cite{Lamb2003,Kukowski2006}.
 \item Heuret and Lallemand \cite{Heuret2005} emphasize the eminent role of acceleration of the westward movement of the upper, South American plate.
Sobolev et.~al. \cite{Sobolev2006} find out that the accelerating westward movement of the South American plate is the most important factor for the Andean orogeny. 
\end{dlist}
The third hypothesis generates the question of the mechanism which drives South America westward. 
The ridge push at the mid-Atlantic ridge and the slab pull at the Lesser Antilles and the Scotia arc have been proposed but it could be that these contributions are too small. 
Schellart and Rawlinson \cite{Schellart2010} remarkably mention the slab pull of the Nazca plate.
However, we propose that a downwelling of the bulk convection beneath South America caused by the thermal screening of the thick continental lithosphere could play a role. 
If there is a large upwelling east of South America and a large downwelling of the bulk convection under South America we could understand why the bulk convection current would move the South American plate westward since, because of the thick continental lithosphere, South America is \emph{not} decoupled from the bulk convection by the asthenosphere. 
Additionally, we observe a large upwelling of the bulk convection beneath the Pacific.
So we can expect that another current of the bulk convection will go eastward, producing the arcs of the Lesser Antilles and Scotia.
It is unclear if these speculative arguments are appropriate.
However, they show not only the necessity of dynamic, numerical regional models but also that the regional model must be nestled among time dependent boundary conditions determined by a global dynamic model.
Evidently, some \emph{essential} features of the Andean orogenesis cannot be explained by an isolated regional model.

We discuss further \emph{new} geological papers in \ref{sec:sched} b).
We did that on purpose in order to substantiate why we used certain details in our development of two special (alternative) mechanisms for a regional model of the South American subduction zone and Andean orogenesis.
Further \emph{new} papers are mentioned in \ref{sec:sched} c) in order to select an appropriate set of prescribed plate movements in the surrounding spherical-shell convection model which is necessary to embed the regional model.

\subsection{Preliminary work}
\label{sec:prelim}
Under a), b) and c), we describe what we have done in the last three years with respect to the project.
Under d), other efforts of us which have some relation to the topic are reported. 
There is a direct relationship between the subsections a), b), c) and the projected works of this application for the coming three years.
In \ref{sec:sched} b) we describe in rather distinct outlines what we have prepared from the geophysical and geological point of view, i.e. we discuss the conception of our specific South American dynamical models which are embedded into the second spherical-shell model.
\paragraph{a) Spherical-shell model: Forward model.}
%%%  FIG 1  %%%
\begin{SCfigure}[1][htb!]
\centering
\includegraphics[width=0.78\textwidth]{fig1_ft8dfg.498.eps}
\caption{Juvenile additions to the sum of continental masses acc. to the \emph{new} convection-differentiation model.
 Run 498, \cite{Walzer2012b}.}
 \label{fig:1}
\end{SCfigure}
%%%  FIG 2  %%%
\begin{SCfigure}[1][htb!]
%\centering
\includegraphics[width=0.73\textwidth]{fig2_cplot0m.498.11.eps}
\caption{The distribution of continents (red), oceanic lithosphere (yellow) and oceanic plateaus (black dots) for the present time according to the \emph{new} convection-differentiation model.
 \cite{Walzer2012b}, Run 498, $r_n=0.5$, $\sigma_y=120$~MPa, continental percentage = 42.2\%.}
 \label{fig:2}
\end{SCfigure}
We developed two spherical-shell convection models. 
The forward model is described here; another model with prescribed plate movements is outlined under \ref{sec:sched} c). 
For both spherical-shell models as well as for the regional models of the Andes (\ref{sec:sched} b), we newly derived, in some cases adopted new radial profiles of the relevant physical parameters of the mantle \cite{Walzer2012b}. 
As a Gr\"uneisen parameter, we calculated an extended acoustic gamma, $\gamma_{ax}$, using seismic observations.
These observed values are the bulk modulus $K$, the shear modulus $\mu$, $dK / dP$ and $d\mu / dP$, where $P$ is the pressure.
This procedure has the advantage to be based on observable quantities without any assumptions on mineralogy.
A further calculated property is the adiabatic gradient. 
We derived new profiles for the thermal expansivity, $\alpha$, and the specific heat, $c_p$, at constant pressure.
For the chemical differentiation we used a simple melting criterion, $T > f_3 \cdot T_m$, where $T$ is the temperature and $T_m$ is the solidus.
We estimated the lower-mantle solidus using values of \cite{Litasov2002} and of other authors continuing them by using our $\gamma_{ax}$.
In the upper mantle and the transition layer, we took into consideration the water dependence of the solidus.
Litasov \cite{Litasov2011} investigated the influence of different water concentrations on the solidus of peridotite and we took the derived depressions of the solidi into account in the computation of our convection-differentiation model of the mantle's evolution.
So in our new model, the melting temperature is a function of time and position. 
The most important innovation of the new convection model concerns our newly derived viscosity profile which is based on solid-state physics and seismological results \cite{Walzer2012,Walzer2012b}.
This viscosity distribution has a resemblance to the viscosity model of \cite{Mitrovica2004} although its derivation is totally different.
The full set of convection-differentiation equations has been solved using the improved code Terra (see b)).
For each run, we obtained lots of parameters, some of which can be compared with observational quantities. 
For ages from 4490~Ma to the present time, we received the curves of the laterally averaged heat flow density $qob$ at the surface, the converted continent-tracer mass per Ma, the Urey number $Ur$, the Rayleigh number $Ra$, the Nusselt number $Nu$, the volumetrically averaged mean temperature $Tmean$ of the mantle, showing a very realistic temperature drop when compared with Archean komatiite temperatures, the integrated mass of continents, the kinetic energy of the mantle $Ekin$, the radiogenic heat production $Qbar$ and the laterally averaged heat flow density $qcmb$ at the CMB.
Further results are the vector field of the creeping velocity and the temperature distribution for every time step of the Earth's evolution.
Up to now, we varied the parameter $f_3$ and the thermal conductivity $k$.
There are $f_3$-$k$ clusters of realistic totalities of solutions. 
Fig. \ref{fig:1} represents an episodic distribution of juvenile additions to the continents which is connected with the episodicity of orogenetic epochs.
Fig. \ref{fig:2} shows the present-day distribution of continents of the same run.
In this example, 42.2\% of the Earth's surface is covered by continents. 
The present-day surface heat flow of this run is $81.39~mW/m^2$, comparable with the observed value of $90.185~mW/m^2$ \cite{Davies2010}.
The present-day value of $qcmb$ of this run is $20.20~mW/m^2$ which seems to be realistic, too.
We have prepared the pictures and tables for a new paper on the \emph{results of a new spherical-shell convection-evolution model} \cite{Walzer2012b}, whereas for the paper on the physical background \cite{Walzer2012} some runs are necessary, yet.

\paragraph{b) Numerical Improvements.}
Two submitted publications of our group \cite{Koestler2012,Mueller2012} deal with \emph{numerical improvements and are to be found in the Appendix.}
Their conclusions will not be discussed here.
However, their results are very important for the realization of our Andean model (\ref{sec:sched} b).

In the last three years, an essential part of our efforts was concentrated on the creation of an essentially improved Terra code, which is necessary to resolve the numerical problems with the newly derived viscosity profiles.
We equally apply this new viscosity profile to a spherical-shell mantle convection model \cite{Walzer2012,Walzer2012b} as well as to the future regional model of the South American subduction slabs (\ref{sec:sched} b).
In 2009, our team started collaborating with other of Terra-developers from Cardiff University and Munich University and intensified the collaboration with J. Baumgardner, San Diego, USA. 
At a first joint meeting in Munich in 2009, the group decided to set up a community svn-repository for further code development, supplemented by trac, a web-based project management and bug-tracking tool, and automated compile/test cycles using BuildBot. 
From then on the group worked on a common code base using automated tests for every revision of the code.
There have been the two successive joint meetings in Cardiff in 2010 and in Jena in 2011 (see also http://www.igw.uni-jena.de/geodyn/terra2011.html).
The progress being made since the first meeting includes the following items. 
\begin{dlist}
  \item Enhancement of the code to increase global resolution and maximum number of MPI processes.
  \item Further development and integration of the Ruby test framework \cite{Mueller2008} into the automated BuildBot tests.
  \item Implementation of a finite-element inf-sup stabilization using pressure-polynomial projections proposed in \cite{Dohrmann2004}.
  \item Development and implementation of an efficient preconditioner for the variable-viscosity Stokes system \cite{Koestler2011}.
  \item Refinement of the Pressure Correction algorithm \cite{Koestler2011}, giving more robust convergence.
  \item Restructuring of the code to use language features of Fortran95 and Fortran2003 where possible.
  \item Integration of automated code documentation using doxygen.
  \item Integration of VTK-support and automated visualization.
  \item Significant improvements in the formulation of the free-slip boundary condition on the spherical surface.
\end{dlist}
Continued effort is spent on several numerical and technical topics as well as on including more realistic physical models. 
In the following, some important details are given.
\begin{dlist}
  \item 
In \cite{Walzer2008a,Walzer2012,Walzer2012b}, two pressure- and temperature-dependent viscosity profiles of the Earth's mantle are developed and used for the computation of two mantle convection models with chemical differentiation of oceanic plateaus and generation of continents.
The new viscosity model includes very strong viscosity gradients at lithosphere-asthenosphere boundary and at 410-km and 660-km phase boundaries.
Therefore, it is very important to make the code Terra fit for such a strong challenge. 
Regarding a physically consistent variable viscosity momentum operator, J. Baumgardner, P. Bollada and C. K\"ostler figured out in which way the code has to be changed to apply a physically consistent A-operator using cell-averaged viscosities. 
The most significant code change is the switch from nodal based to triangle based operator parts on the sphere. 
The viscosity-weighted summation over triangular integrals is then done in the application of the operator. 
We expect that the cost for applying $Au$ will be doubled but a consistent formulation on all grid-levels could pay off for this, especially if we get a better convergence rate of the multigrid algorithm.
The implementation of the triangle-based operator formulation is now under way. 
 \item Exporting Terra-operators in sparse matrix format:
When exporting the FE-matrices for the whole grid, they can be analyzed and PETSc or other parallel solver packages can be applied to it. 
This is intended to provide flexibility and to ensure reliability of future code changes.
 \item Multigrid(MG)-implementation: 
As mentioned in \cite{Tackley2008}, the current MG-implementation in Terra does not fulfill the expectations raised by the performance of a 2-D Cartesian version of Terra, documented in \cite{Yang2000}. 
In \cite{Koestler2011} it is also identified to be the worst performing part of the iterative solver in Terra. 
It is only poorly analyzed, and it is not satisfactory documented. 
M. Mohr is going to analyze and document the currently used matrix-dependent transfer multigrid in detail.
The MG-implementation is also to be changed to using cell-based viscosity averages.
 \item Free-slip boundary condition and propagator matrix benchmark tests:
P. Bollada and R. Davies showed that adding boundary terms to the right hand side of the momentum equation reduces some sort of errors while other kinds of errors still exist.
They will continue to figure out the exact cause of that behavior and work on fixing this. 
With direct access to the radial velocity component, the local spherical coordinate system offers a way to straightforward implement the free-slip boundary condition. 
J. Baumgardner implemented this in a local copy of Terra, and it is ready to be used. 
The group has agreed to create a repository branch to continue working on that. 
If successful, the local spherical coordinate system-version will be merged into the trunk after some months of testing.
 \item Adding Ruby tests: 
The Ruby test framework is ready to be used extensively in testing Terra's subroutines as individually as possible.
It can also be used for debugging by application of subroutines to predefined scalar and vector fields.
Still the test coverage of the Terra by Ruby tests is very low and needs to be extended.
\end{dlist}

\paragraph{c) Andean model.}
The largest expenditure of time of U. Walzer in the last three years was the analysis and synopsis of geophysics, geology and geochemistry of the Andean orogenies which has been done in close cooperation with J. Kley and L. Viereck-G\"otte.
An extensive model instruction has been given to C. K\"ostler on 31 January 2011. 
An earlier delivery would have been senseless before the successful solution of the described numerical problems (cf. b)).
Some geological and geophysical considerations behind this instruction are outlined in \ref{sec:sched} b).

\paragraph{d) Long-term related works.}
J.~Kley investigated the structural geology of some areas of the central Andes \cite{Kley1993} and gave a regional structural analysis and kinematic restoration \cite{Kley1996,Kley1994}. 
The FU~Berlin offered an excellent environment for integration of geological and geophysical results, ranging from a regional scale \cite{Kley1996a} to that of almost the entire orogen \cite{Schmitz1997}. 
In the SFB 267, Kley participated in an effort to extend quantitative structural analysis to a transect right across the backarc area. 
First steps were also taken towards constraining the evolution of strain rates over time \cite{Kley1997}, leading to the suggestion that the continental strain rates had increased during the Andean orogeny. 
J.~Kley began to extend kinematic analysis to the entire orogen, employing serial balanced sections and map view restoration techniques \cite{Kley1998a}. 
A map view kinematic model of the central Andes \cite{Kley1999} formed the basis for comparison of geologically derived, orogen-scale kinematics with GPS and seismologic data 
%in a cooperation with GFZ Potsdam and the University of Northern Illinois 
\cite{Hindle2002,Hindle2002a,Klosko2002}. 
It could be shown that the present-day strain field from satellite geodetic data closely matches the strain field for the last 10 Ma as inferred from geologic evidence. 
Additional evidence was presented that the strain rate in the South American plate becomes independent of plate convergence rate in the later stages.% of the Andean evolution. 
Using the map-view strain field as input for a numerical model, an attempt was also made 
to constrain crustal thickness evolution and the flux of crustal material 
during the Andean orogeny \cite{Hindle2005}. The studies on variations in structural style 
along the Andes \cite{Kley1998} also triggered a second line of research 
dealing with the influence of inherited lithospheric heterogeneities 
on the spatial strain distribution. One important factor here is the widespread occurrence 
of Mesozoic rift basins \cite{Kley1999b} that were partially inverted as the thrust front 
migrated across them. Several case studies from a particularly well-exposed rift system 
in northern Argentina helped to clarify the importance of fault reactivation 
and stratigraphic discontinuities in conditioning the mechanical behavior 
of the upper crust in contraction \cite{Kley1999a,Kley2002,Kley2005,Monaldi2008}.
The results of these studies were incorporated in \cite{Oncken2006,Kley2007}. 

U. Walzer and cooperators worked on convection-fractionation problems. 
The thermal evolution of the mantle and the chemical evolution of the principal geochemical reservoirs have been modeled simultaneously by a fractionation mechanism plus 2D-FD thermal convection \cite{Walzer1997,Walzer1997a,Walzer1999,Walzer2000}.
Oceanic plateaus, enriched in incompatible elements, develop leaving behind
the depleted parts of the mantle. The resulting inhomogeneous heat-source
distribution generates a first feed-back mechanism. The lateral movability 
of the growing continents causes a second feed-back mechanism \cite{Walzer1999}.
Effects of the viscosity stratification on convection and thermal evolution
of a 3D spherical-shell model have been investigated and a 
viscosity profile of the mantle was developed \cite{Walzer2004,Walzer2004a}. 
The paper \cite{Walzer2005}
presents 2D and 3D thermochemical models of mantle evolution where a 
self-consistent theory is included using the Helmholtz free energy, 
the Ullmann-Pan'kov equation of state, the free-volume Gr\"uneisen parameter
and Gilvarry's formulation of Lindemann's law. In order to obtain the relative
variations of the radial factor of the shear viscosity, the pressure, $P$, 
the bulk modulus, $K$, and $\partial K / \partial P$ from the seismic model
PREM have been used. The publications \cite{Walzer2008a,Walzer2004a,Walzer2004b,
Walzer2006}
present models of self-consistent generation of stable, but time-dependent plate
tectonics on a 3D spherical shell. 
Different types of solutions have been found for different models by systematic variation of parameters 
\cite{Walzer2003,Walzer2004a,Walzer2007,Walzer2008c,Walzer2008a}. 
Stirring effects are investigated in \cite{Gottschaldt2006}. A 3D spherical-shell 
mantle convection and evolution model with growing continents 
\cite{Walzer2007,Walzer2008,Walzer2008a} has been developed. The evolution model
equations guarantee conservation of mass, momentum, energy, angular momentum, 
and of four sums of the numbers of atoms of the pairs $^{238}$U-$^{206}$Pb, 
$^{235}$U-$^{207}$Pb, $^{232}$Th-$^{208}$Pb, and $^{40}$K-$^{40}$Ar.
The pressure- and temperature-dependent viscosity is supplemented by a 
viscoplastic yield stress. The lithospheric viscosity is partly imposed to
mimic the viscosity increase by chemical layering and devolatilization.
Stochastic effects \cite{Walzer2008b} are shown to exist especially in the
chemical differentiation. Although the convective flow patterns and the
chemical differentiation of the oceanic plateaus are coupled, the evolution
of the time-dependent Rayleigh number, $Ra_t$, is relatively well predictable
from run to run and the stochastic parts of the $Ra_t(t)$~-~curves are small
\cite{Walzer2008a}. Paper \cite{Walzer2008} deals with connections between
plate tectonics and the conditions of the existence of life. 

L.~Viereck-G. and cooperators worked not only about the genesis of the Jurassic Ferrar large igneous province in Antarctica but also about the connection between Antarctica, South America and Africa. 
Plateau forming lavas in the Karoo province of South Africa and in the Ferrar province in Antarctica were emplaced synchronously at approx. 180~Ma. 
They thus seem to originate from the same dynamic mantle process. 
However, both are distinguished by their isotopic characteristics with respect to the Rb/Sr- and Sm/Nd systems: while the Karoo magmas show mantle values, the Ferrar magmas exhibit enriched upper crustal values. 
However, the boundary between both Jurassic ig\-ne\-ous provinces is marked by a large transpressional shear zone (Heimefront SZ) of Pan-African age (600-500~Ma), crossing Dronning Maud Land (S-African side of Antarctica) on the continent side of the Grunehogna craton, a frag\-ment of the W-Gondwana Transvaal craton. 
This shear zone is interpreted as a reactivated suture of Grenvillean age (1.1~Ga). 
If Jurassic magmas on either side of this boundary are isotopically different, it must be concluded that \\
(1) this is not a signature in a lower mantle plume, \\
(2) this must be a signature within the subcontinental lithospheric mantle, \\
(3) this signature must be older than Grenvillean in age, \\
(4) it must be introduced into a paleo supra-subduction mantle wedge by subduction processes if it is a crustal isotopic signature.\\
Their studies concentrated on the timing of the initiation of the Ferrar as a large igneous province as well as on the physicochemical characterization of the melt source region conditions during melt differentiation. 
Studying the intrusive, extrusive and volcaniclastic rocks:
\begin{dlist}
 \item They reconstructed the initiation of a large igneous province to have occurred in several steps of melt pulses within 5~Mio years ($<$189 to 183~Ma ago).
 \item They showed initiation to have started with large volume shallow level intrusions of low-Ti andesitic melts into wet fluvial sediments (Triassic/Jurassic in age) associated with diatreme-forming Taalian-type (syn. non-marine Surtseyan-type) eruptions.
 \item The eruptions followed by basaltic andesites in small volume eruptions of partly pillowed lavas from local eruptive centers prior to large volume plateau forming lava extrusions from feeder dikes
 \item followed by a final pulse of a large volume andesites high in Ti that had differentiated from a common primary melt under lower pressure, oxygen fugacity and water activity.
\end{dlist}
All melts belong to the tholeiitic differentiation series, however with orthopyroxene instead of olivine as early fractionating mafic solidus phase. 
REE, Sr-Nd-isotope and PGE characteristics indicate generation of the primary melts within the spinell-lherzolite zone of an isotopically enriched and sulfur-undersaturated subcontinental lithospheric mantle. 
Crustal silica enrichment of this source due to an overprint in a supra-subduction environment had already been concluded from Re-Os-isotopy. 
Our Sr-isotope data in plagioclase phenocrysts, however, exhibit decreasing radio\-genic character in subsequent melt pulses indicating additional assimilation of crust to decreasing extents during ascent and differentiation.
Due to a Cretaceous thermal event, $^{40}$Ar/$^{39}$Ar-ages (ranging from 235~Ma to 90~Ma) and S-isotope characteristics are heavily disturbed, only a few samples exhibit the relict primary \mbox{$\delta^{34}$S-value} of -19,5.

%% ------------------------------------------------------------------------ %%
%
%  SECTION 3
%
%% ------------------------------------------------------------------------ %%

\section{Objectives and work schedule}\label{sec:3}
\subsection{Objectives}
First, there will be some fundamental remarks. Our modeling of the Andean slab should be compatible with our basic assumption that plate tectonics is an integral part of the convection of the entire mantle.
Therefore, we should aim at a regional model which is embedded into a realistic convective spherical-shell model. 
It is possible that the mantle convection is characterized by top-down control. 
But it is also possible that the larger part of the mantle mass essentially determines the movements at the surface since the most important share of the primordial energy is stored there and also the principal share of the radiogenic energy is released there.
Here we refer to the absolute value, not to the density of these quantities. 
On the other hand, for reasons of energy it is evident that the Earth's core cannot play a prominent part in controlling mantle convection.
It is exactly the reverse. 
The mantle convection determines the boundary conditions of the hydromagnetic convection in the outer core.
E.g., in time spans of high activity of mantle convection, the latter one generates lateral temperature differences in D'' which reduce the number of magnetic reversals or even make them impossible because of the anti-dynamo theorems.

Furthermore, it is well known at the present time that the oceanic lithosphere consists of three (or more) layers which are chemically different and that its lower boundary is characterized by a sharp viscosity jump.
At the same time the oceanic lithosphere is also a thermal boundary layer. 
So, we do not intend to use the old simplified approach that the existence of an oceanic lithosphere can \emph{exclusively} be explained by a thermal boundary layer.
In this case we should expect a gradual decrease of the viscosity at the lower boundary. 
Furthermore, we conclude that the principal part of the buoyancy of the slab heavily depends on chemistry and that phase transitions, especially the basalt-eclogite transition, play an eminent role.

\emph{So, we want to combine a global 3-D spherical-shell convection model with a 3-D regional convection model of South America and the surrounding plates and a model of the orogeny of the Cordilleras.}
The Andes are an ideal test case for various reasons:
\begin{dlist}
 \item They are very large, thus keeping the inevitable problems of different scales at a minimum. 
 \item They are active and well-studied. A wealth of information constrains the processes of orogenesis.
 \item They have a simple large-scale geometry. The plate margin is gently curved and was even straighter in the past. Convergence is orthogonal to the mean trend of the margin. 
The age structure of the Nazca plate is symmetric, with the oceanic lithosphere oldest in the center and younging to both sides. 
 \item Their plate kinematic framework has remained nearly unchanged for the last 50~Ma. 
On the other hand, there was a prominent switch in the mode of subduction earlier, from presumably steep with backarc extension to low-angle with backarc shortening. 
In the Andes, both end-member settings can therefore be studied in the same place. 
 \item There is a clear compositional and rheological contrast between the two converging plates. 
It is unlikely that large volumes of material are transferred across the plate contact, in strong contrast to continental collision zones. 
 \item The Andean substrate has a simple geologic history. 
Much of the South American plate was in place before subduction started some 200~Ma ago. 
Except for the northern Andes, no terranes were accreted after the Paleozoic.
\end{dlist}

We aim at a numerical, nearly purely dynamic model with a minimum number of restrictions and additional assumptions. 
This model is to explain the physical mechanism of the essential features of Andean orogeny. 
We gathered experience in structural geology of the Andes and in the numerical geodynamics. 
Therefore, we expect that the project will result in appreciable synergetic effects. 
There is a large number of unexplained or partially unexplained geological and geophysical observations, e.g.:\\
(a) Why do we observe at some places flat-slab subduction, at other places normal subduction although the neighboring plate velocities are not extremely different? \\
(b) How can we explain certain geochemical episodicities? (See \ref{sec:sched} b.)\\ 
(c) Why is the mountain building of the Andes \emph{episodic} in spite of essentially \emph{continuous} subduction?\\ 
(d) What is the reason for the general trend of migration of the shortening zone from West to East?\\
(e) What is the main mechanism to create an orogenetically active backarc zone?\\ 
Referring to the method we want to go off constructing indenter models.
Therefore, we propose a regional model embedded into a global convection model with prescribed plate movements outside the investigated region. 
We developed a detailed conception how to devise the model.
We partly present this conception in \ref{sec:sched}. 


%3.2 Work schedule
\subsection{Work schedule}
\label{sec:sched}

\paragraph{a) Outlines of the model.}
We do not intend to develop an exclusively regional model of the Andean orogenesis since, in this case, the temporally varying boundary conditions are unknown. 
Therefore, it is often assumed, for reasons of simplicity, that there are no effects from outside of the regional computational domain. 
However, some changes in the arc volcanism and in the tectonic shortening of the Andes suggest a connection with the 30-Ma-Africa-Eurasia collision. 
Therefore, we intend to embed a regional 3D model into a 3D spherical-shell model.
So, we want to solve the balance equations of momentum, energy and mass in the spherical-shell model using somewhat larger time steps on a coarser whole-mantle grid, coarser than in the regional model. 
The values of creeping velocity, temperature and pressure, determined in that way and lying at the boundaries of the {\em regional\/} computational domain, then serve as boundary conditions for a computation with smaller time steps for which the balance equations are solved in the regional computational domain. 

\vspace{-2.5ex}
\paragraph{b) The regional model.}
To design the regional model, we developed not only extensive numerical improvements (see \ref{sec:prelim}) in the Terra code but studied also the latest geophysical, geological and geochemical results taking them into consideration in the draft of a physically reliable mechanism which is not only geodynamically probable but also numerically feasible (U.~Walzer, J.~Kley, L.~Viereck-G\"otte).
There are several geologically descriptive model proposals. We mention only \cite{Gutscher2000,Haschke2006,Oncken2006,Ramos2009,Kay2009,DeCelles2009}.
Our designed computable model system ought to solve the following problems.
\begin{dlist}
 \item Why do we observe today in some segments of the Andes flat subduction and magmatic lull \{Bucaramanga, Peruvian (2\textdegree~S to 15\textdegree~S) and Pampean (27\textdegree~S to 33\textdegree~S) flat slab \cite{Ramos2002}\}, but in other segments normal subduction with dip angles between 30\textdegree and 40\textdegree ?
This is amazing since the westward velocity of the South American plate and the eastward velocity of the Nazca plate do not \emph{essentially} vary alongshore.
 \item Why do the flat-slab segments migrate in Cenozoic times along-side the Andes \cite{Ramos2009}?
Ramos and Folguera \cite{Ramos2009a} found an almost continuous belt of flat slabs which migrate southward.
 \item Why is the present-time volcanism restricted to segments with a 30\textdegree{} to 40\textdegree{} dipping slab \cite{Sebrier1991,Kay1996,Allmendinger1997,Kley2007}?
 \item Why does the deformation essentially start in the West and migrate to and finish in the East \cite{Oncken2006}?
We derived some estimations on some other hypotheses. 
After that we consider the assumption as most promising to answer the mentioned questions by the assumption that oceanic plateaus and aseismic ridges are carried by the conveyor belt, i.e. by the Nazca plate.
The plateaus and ridges generate additional positive buoyancy \cite{Gutscher2002}. 
So the hinge of the slab migrates to the East until the volcanism totally vanishes.
However, if we scrutinize a good geological map of South America we notice that the relation between flat-slab segments and ridges are not simple.
We can assign the Pampean flat slab to the Juan Fernandez Rigde \cite{Alvarado2009}.
Even the southward migration of the Pampean flat-slab zone can be explained by form and movement of the Juan Fernandez Rigde \cite{Kay2009}.
The southern part of the Peruvian flat slab can be referred to the Nazca Rigde. 
For the northern part of the Peruvian flat slab we have to introduce the hypothesis of an immersed Inca Plateau \cite{Gutscher2000}.
However, east of the Carnegie Rigde there is an abundant volcanism in Ecuador.
Michaud et al. \cite{Michaud2009} show a detailed reconstruction of the eastward movement of the Carnegie Rigde which extends at least 60~km below the South American plate with a continuous plunging slab down to a depth of 200~km.
The adakitic signal is proposed to be ridge-induced.
In the case of the Iquique rigde we have to assume that there is no eastern continuation of it. 
Pindell and Kennan \cite{Pindell2009} show that the flat-slab area in northern Colombia and Venezuela might be considerably larger than the area assumed by Ramos \cite{Ramos2009}.
 \item \emph{Why should our designed mechanism of the generation of the Andes be 3-D?} 
Several observations suggest the idea that essential features of the Andean orogenesis cannot be explained by 2-D dynamical models:\\
a) Hindle et~al. \cite{Hindle2005} conclude that a mass balance which is restricted to a cross-section through the Andes leads into contradictions.
They show that there has to exist an essential mass transport alongside the Andes. 
In particular, the displacement of material toward the axis of the bend in the central Andes leads to a significant crustal thickening. 
This cannot be explained by a two-dimensional model, neither kinematically nor dynamically.\\
b) Anderson et~al. \cite{Anderson2004}, Kneller and van Keken \cite{Kneller2007} and Barnes and Ehlers \cite{Barnes2009} show and discuss, for the southern Andean subduction zones, trench-parallel high seismic shear velocities near to the trench and an abrupt transition to trench-perpendicular high seismic shear velocities in the back arc. 
The Brazilian subcrustal lithosphere beneath the eastern Cordillera, the Interandean and the Subandes are East-West fast whereas the shear-wave velocity under the Altiplano and Puna has maximum values in the North-South direction.
This is a hint that a significant three-dimensional flow might be involved in the mechanism.\\
%%%  FIG 3  %%%
\begin{figure}[htb!]
\begin{minipage}[b]{0.65\textwidth}
  \centering
  \includegraphics[width=0.99\textwidth]{fig3_Andes_391.eps}
  \caption{The geometric starting configuration in our second Andean regional model, taken from \cite{Vietor2006}.}
  \label{fig:3}
\end{minipage} 
\hspace{0.02\textwidth}
%%%  FIG 4  %%%
\begin{minipage}[b]{0.33\textwidth}
  \centering
  \includegraphics[width=0.99\textwidth]{fig4_grid_jb.eps}
  %\caption{The diamond (red), embedded in the global spherical grid, derived from a sequence of dyadic refinements of the projection of the regular icosahedron onto the sphere, modified from \cite[Fig. 1]{Baumgardner1983}.}
  \caption{The diamond (red), embedded in the global grid, which will be dyadically refined to the model resolution. Modified from \cite[Fig. 1]{Baumgardner1985}.}
  \label{fig:4}
\end{minipage} 
\end{figure}
c) \emph{The toroidal component.}
A pertinent argument for the necessity of 3-D models results from the following considerations and calculations. Already Gable et~al. \cite{Gable1991} and O'Connell et~al. \cite{OConnell1991} showed the relevance of the toroidal-poloidal partitioning for lithospheric plate motions. 
The lateral subducting slab movement induces slab-parallel flows and a rollback-generated flow around the slab \cite{Schellart2004}. 
Stegman et~al. \cite{Stegman2006} demonstrated that in the case of non-vanishing rollback of the subduction slab for some typical cases, 69\% of the energy of the negative buoyancy of the slab is converted into the toroidal component of the rollback-induced flow whereas 18\% are consumed for the weakening of the plate.
These numbers show the importance of a 3-D modeling in a striking way.
Only in the very beginning of the Andean-specific modeling we will prescribe the velocity of the migrating subduction hinges as a function of time according to \cite{Oncken2006} in order not to overburden the model.
But the other degrees of freedom of the flexible slab should be determined by the differential equations of the model.
We should vary the lateral extend of the individual slab between 200 and 5000~km. 
That is, in a first type of numerical experiments we will introduce the individual lobes shown by the distribution of seismicity.
In a second numerical experiment we will assume an undivided slab for the whole South American continent (Fig. \ref{fig:3}), at best with a disconnection at the Chile Rise.
For all versions we cut out a spherical diamond (Fig. \ref{fig:4}) from the newly improved and inf-sup stable Terra code which is able to solve the convection differential equations in a spherical shell mantle.
Essential parts of the South American plate and the Nazca plate fit into this spherical diamond. 
The vertical boundary conditions are iteratively taken from a whole-mantle convection model. Cf. \ref{sec:sched} c).
In contrast with \cite{Stegman2006}, we do not neglect the energy equation.
It is evident that the subduction mechanism is possible only assuming a low-viscosity asthenosphere \cite{Fischer2010} which is less dense than the average oceanic lithosphere. 
Furthermore, the rheology may not be purely viscous. 
For this purpose we \cite{Walzer2008a} prefer a viscoplastic yield stress.
The lithospheric-asthenospheric boundary for the viscosity is relatively sharp, also for the oceanic lithosphere. 
This is a special challenge for our code.
%%%  FIG 5  %%%
\begin{SCfigure}[1][htb!]
\centering
\includegraphics[width=0.77\textwidth]{fig5_Andes_343.eps}
\caption{The La/Yb ratio as a function of time for the igneous rocks of the north Chilean arc (21-26\textdegree~S) acc. to Haschke et.~al. \cite{Haschke2006}.
Note that \emph{after} the flat-slab stage, the La/Yb \emph{suddenly} decreases to a low starting value.
The periods of orogenetic activity are \emph{before} these drops.}
 \label{fig:5}
\end{SCfigure}
 \item \emph{By the model, it should be possible to understand essential geochemical observations relevant for South America.}
The difficult problem of delamination is very probably connected with this issue.
It is necessary to clarify what kind of delamination is dominant. 
Fig. \ref{fig:5} shows the gradual increase of the flow of incompatible elements.
After each flat-slab period, this rise is abruptly interrupted. 
Such kind of plots exist not only for the mass ratio La/Yb but also for $^{87}$Sr/$^{86}$Sr and $^{144}$Nd/$^{143}$Nd.
These curves describe a slow growth of the abundances of elements with large ionic radii within one cycle since $^{87}$Sr is the daughter isotope of $^{87}$Rb, $^{143}$Nd is the daughter isotope of $^{147}$Sm and the other mentioned isotopes are stable. 
The fluctuations of the whole-rock initial $\epsilon_{Nd}$ of the central Andean arc by \cite{DeCelles2009} are evidently related to Fig. \ref{fig:5}. 
DeCelles et~al. \cite{DeCelles2009} emphasize that the cyclical changes of the isotopic compositions of arc magmas cannot be explained by changes in the convergence rates. 
They and also we expect that these changes have to be explained by episodic gravitational foundering. There are two principal possibilities.\\
\textbf{A)} According to \cite{DeCelles2009}, below arc and hinterland, i.e. below the western parts of the South American continent, the eclogitization of the thickening lower continental crust and of the lithosphere mantle causes a density increase and therefore a delamination so that these units sink into the mantle wedge driven by their own weight. 
Carlson et~al. \cite{Carlson2005} discuss the possibility that the continental mantle above the wedge of the mantle overlying a subducting oceanic plate can become unstable.
The detachment from the overlying continental crust can cause major orogenetic episodes. 
Davidson and Arculus \cite{Davidson2006} propose a delamination of the cumulate layers below the seismological Moho back into the mantle of the sub-arc wedge.
The two-dimensional numerical model by Sobolev et~al. \cite{Sobolev2005,Sobolev2006} is compatible with the geological models mentioned under a).
They use a viscoelastic rheology supplemented by Mohr-Coulomb plasticity for the layered lithospheres. 
The drift of the overriding plate and the pulling of the slab is prescribed by the velocities at the boundaries of the 2-D model area and it is not calculated by solution of the balance equations though. 
What drives Andean orogeny? 
Sobolev et~al. \cite{Sobolev2005,Sobolev2006} answer this question by numerical experiments using their 2-D model and varying only one influence parameter each.
They conclude that the major factor is the westward drift of the South American plate. 
Paragraph a) outlines only \emph{one} way of thinking which we intend to test.\\
\textbf{B)} Here we propose a second hypothesis to understand the mentioned geochemical observations of \cite{Haschke2006} and of \cite[Fig. 5]{DeCelles2009}. 
This hypothesis is patronized by the idea of a geochemical marble-cake mantle
\cite{Walzer2008a,Walzer2011} but composed of irregularly formed parts of a depleted mantle (DMM) with \mbox{80-180 ppm H$_2$O} and \mbox{50 ppm C} and another, richer reservoir with \mbox{550-1900 ppm H$_2$O} and \mbox{900 - 3700 ppm C} \cite{Wood1995,Green1998,Wood2007}.
It is possible that the reservoirs are intermixed in smaller quantities so that there is no sharp boundary between the different parts of the mantle \cite{Hofmann2003,Stracke2005}.
DMM dominates in the smaller depths of the mantle.
The deeper the slab dives into the mantle, the higher is the probability to touch regions with a high abundance of $^{3}$He and incompatible elements.
Pilz \cite{Pilz2008} investigated  $^{3}$He/$^{4}$He ratios in the Puna plateau and at the volcano Tuzgle. He found that these $^{3}$He/$^{4}$He values are higher than in the western Cordillera and in the Salta Basin east of it.
Pilz and also we conclude that a provenance by degassing of the slab is not feasible since the mantle's $^{3}$He is primordial.
The $^{3}$He of the atmosphere cannot be subducted in appreciable quantities. -- 
There are different, but related models of chemical layering inside the subducting oceanic lithosphere \cite{Ruepke2004,Ganguly2009}.
There is a water-rich subduction channel above the slab.
Because of the sediment dehydration and the crustal dehydration it is not entirely clear how deep this hydrous channel is extended.
Not only the $^{3}$He/$^{4}$He ratio of the Puna plateau \cite{Pilz2008} but also La/Yb, $^{87}$Sr/$^{86}$Sr and $^{144}$Nd/$^{143}$Nd increase from an age of $\tau = 4.5$~Ma until the present time \cite{Haschke2006}.
There are different explanations for this phenomenon. 
One of them would be a rise in an antiparallel direction in hot fingers immediately or in some distance above the subducting slab.
Such a suggestion has been offered by \cite{Tamura2002} for the NE Japanese arc.
Marsh \cite{Marsh2007} proposed a similar idea of a hydrothermal flow field, in this case immediately above the upper surface of the down-going slab.
In this way he explains also the sharp line of volcanoes or the volcanic front which can be observed in the Andes.
Pilz \cite{Pilz2008} concludes from seismic observations that such hot fingers are also in the wedge of the southern central Andes and that these fingers are near the surface of the slab.
So we want to develop a numerical model with fluid pathways of hydrothermal fluids, described by a particle approach, not very far from the slab but with an antiparallel flow direction.
This idea is corroborated also by \cite{Furukawa2009}.
\end{dlist}

We are working about the numerical problem to cut out a spherical diamond, out of the dynamical 3-D spherical-shell model, and to determine the temporally changing boundary conditions at the vertical side walls of the 3-D diamond by the solution of the convection in a spherical shell, but now with prescribed plates at the surface of the sphere.
The Nazca plate moves with a (in the first approach) prescribed angular velocity through the margin into the computational domain and dives under the South American continent because of a Rayleigh-Taylor instability which is induced \emph{mainly} by the transition to eclogite. 
The oceanic plate has a given sandwich structure of zero-pressure densities, $\rho_0$, \cite{Ganguly2009} and viscosities, $\eta_0$.
The slab is supposed to be freely movable at the surface and floating inside the mantle. 
It is well-known that it is difficult to detach a spherical-shell plate from the surface of the shell into the mantle in a slab-like manner \cite{Becker2009}. 
As in \cite{Walzer2008a}, we want to introduce a viscoplastic yield stress at the near-surface lithospheric region and compare the creeping viscosity, $\eta_c$, at each position and time with the plastic viscosity, $\eta_p$, and use the minimum value in the model.
This procedure is sensible from the physical point of view. 
If a piece of the Nazca plate which is thickened by an oceanic plateau or a passive ridge approaches to South America then it pushes the hinge back under the South American plate because of its positive buoyancy. 
Therefore the volcanic front migrates to the East. 
The residual slab will become steeper and its tip will touch deeper and deeper parts of the mantle.
Therefore, more and more atoms of incompatible elements will rise using the fluid pathways explaining the rising parts of the curve of Fig. \ref{fig:5}.
We are able to describe the movements of the hydrothermal fluid by a tracer modulus. These markers, describing the velocities of higher abundances of incompatible elements, $^{3}$He, metals like Cu, Ag, $\cdots$, are \emph{not} carried along with the creeping rock.
They can move along a surface antiparallel to the slab which represents a thin layer. 
A serpentinized layer-like part of the mantle wedge just above the subducting oceanic crust could serve in reality as such a thin layer \cite{Guillot2009}.
The slab movement is (in this model as in reality) an integral part of the solid-state mantle convection.
The mentioned quantities $\rho_0$ and $\eta_0$ of the slab will be additionally described by other tracers which, in this case, will be entrained and carried along by the solid-state creep down to the interior of the mantle.
The generation and eastward migration of major Andean ore deposits \cite{Kay1999,Kay2001,Rosenbaum2005,Haschke2006}, the eastward migration of the volcanic front, the eastward movement of the deformation ages across the southern central Andes (21\textdegree~S) \cite{Elger2005,Oncken2006} as well as the eastward migration of deformation in the Interandean and Subandean \cite{Kley1996,Ege2004} can be dynamically modeled by our approach. 
When the subducting movement of the down-hanging part of the slab (and the flat-slab part behind) is accelerated, then an elevated activity of the Atacama fault zone, the Peruvian shortening, the Incaic shortening etc. are induced.
The sudden interruption of the supply of highly incompatible elements (Fig. \ref{fig:5}) must be induced by a radical event.
Using the analogous, geologically describing models of \cite[Fig.~4]{Li2007} and \cite[Fig.~7]{Humphreys2009}, it cannot be concluded that many unusual features of the Permian-Jurassic South China fold belt can be explained by shallow subduction with an extensive final foundering of the lower plate combined with a roll-back of a small remnant subducting slab in the Mid-Jurassic.
Humphreys \cite{Humphreys2009} concludes for the Mid-Tertiary that an extensive sinking of the flat Farallon slab occurred which caused an uplifting of the continent covering a large area of the western North America.
We propose an explanation by an extensive generalized eclogitization of the flat-slab part of the oceanic plate and an \emph{extensive} delamination due to a Rayleigh-Taylor instability.
The latter process can be simulated for the former and present South American flat slabs using a particle approach.
The extensive tear-off would entirely interrupt the supplies of incompatible elements, $^{3}$He, etc. since the tracer transport path near the upper surface of the low-hanging part of the slab and the flat part of the slab is ripped.

At first glance, \textbf{A)} and \textbf{B)} seem to be competing models, but in reality they do not entirely exclude one another. 
It is evident that the two roughly sketched numerical regional models are very ambitious. Therefore, we could report here only on the numerical and mathematical results of the preparatory efforts, the further development of the code and our geodynamical conception. However, a spherical-shell model is nearly finished.

\vspace{-2.5ex}
\paragraph{c) Spherical-shell model: Prescribed angular velocities of the lithospheric plates.} 
To define the time-dependent boundary conditions of the vertical side walls of the spherical diamond containing South America and the Nazca plate, we introduce a spherical-shell convection model with the same radial profiles of the relevant physical parameters as in \cite{Walzer2012,Walzer2012b} but with prescribed angular velocities of the plates for the last 200~Ma. This desirable time span is essentially determined by the available observational data, e.g. by Fig. \ref{fig:5}. 
But the most plate-motion models do not go back so far.
However, the main difficulty refers to another item.
Even though we know all velocities of neighboring plates between each other, we do not know the ``absolute'' velocities of the plates relative to the highly viscous mid part of the lower mantle. 
But the net rotation of the averaged lithosphere is important for the kinematic analysis and the dynamic modeling of the slabs, especially the slabs at the margin of South America.
The ``global tectonic map'' of \cite{Schellart2010} is based on the Indo-Atlantic hotspot reference frame by \cite{ONeill2005} and on the relative plate motion model by \cite{DeMets1994}. 
This map shows a large eastward motion of the Nazca plate, large in comparison to the magnitude of the velocity of the South American plate. 
In relation to this feature, this map is similar to the deforming, no-net-rotation reference frame model GSRM by \cite{Kreemer2003}.
In contrast to these two models, the hotspot reference model HS-3  \cite{Gripp2002} shows a high westward velocity of the South American plate, high in comparison with the amount of the velocity vectors of the Nazca plate.
HS-3 is based on the age progressions of ten Pacific ocean islands. Becker \cite{Becker2008} and Becker and Faccenna \cite{Becker2009} wrote a very good analysis of the problem of ``absolute'' plate velocities. 
Becker \cite{Becker2008} and Long and Becker \cite{Long2010} try to determine the present-day plate velocities from the convective shearing movements and the seismic anisotropy of the upper mantle.
But this procedure is more sensitive to the direction of the velocity vector than to its magnitude. 
HS-3 contains a large net rotation of the laterally averaged lithosphere relative to the high-viscosity parts of the lower mantle.
The majority of the authors derived a smaller real net rotation which is defined as the spherical harmonic degree $l=1$ component of the toroidal part of the plate velocities. 
Ricard et~al. \cite{Ricard1991} estimated 30\% of HS-3, Steinberger et~al. \cite{Steinberger2004} calculated 38\%.
The azimuthal seismic anisotropy is compatible only with values less than 50\% of HS-3 \cite{Becker2008}.

Already early, Steinberger and O'Connell \cite{Steinberger1998} linked the hot spot tracks and the movement of oceanic lithospheric plates.
Gurnis et~al. \cite{Gurnis2011} report on a very practicable open-source system which contains the angular velocities for the plates from 140~Ma to the present time. 
Each plate has a time-dependent Euler pole. 
The plates are described by time-dependent closed plate polygons. 
Each of these plate boundaries has their own, time-dependent Euler pole.
The code allows to introduce new interactive plate boundaries \cite{Boyden2011}.
We intend to use two or three seemingly realistic spherical-shell plate-motion models to define the boundary conditions of our regional convective system of South America and its surrounding. 
The repercussions of the different spherical-shell systems on the regional model should be investigated and compared.

\vspace{-2.5ex}
\paragraph{d) Future numerical and technical improvements.}
In addition to the joint work within the group of Terra developers, further developments are to be continued regarding the following items.
\begin{dlist}
\item We are going to further improve the parallelization of the particle tracking routines in Terra. 
Compared to previous Terra versions, there is an extra need for communication among several MPI-processes to figure out connected regions of partial melting in the mantle from which incompatible elements are extracted and transported to the surface. 
A similar communication is required to define the extent of continental lithospheric plates. 
With the high number of tracers, it is crucial to compress the required global information locally before it is exchanged among neighboring processors. 
R. Hendel will continue to reduce the communication overhead for tracking globally connected regions, so that the scalability of the particle routines will be extended to 500 and more processors. 
Such an optimized way of communicating global regions and features is also needed in modeling the elasticity of the subducting plates in the regional Andean model.
\item We also plan to bring the elasticity model in Terra to work. 
It has also to be chosen carefully how elasticity is dealt with in the solution of mass and momentum equations. 
It could be necessary to iterate over the whole Stokes within every time step until an equilibrium between elastic and viscous forces is reached.
\item To run both, regional and global models, with time-dependent boundary conditions at the surface, plate reconstruction data will be imported from the GPlates code \cite{Boyden2011} (www.gplates.org). 
The development of the interface between GPlates and Terra will be done together with L. Quevedo, Sydney.
\item Furthermore,the documentation of the code, which is build from source code comments automatically with doxygen, will be enhanced to make it easier  for new developers to work on Terra.
\end{dlist}



%% ------------------------------------------------------------------------ %%
%
%  SECTION 4
%
%% ------------------------------------------------------------------------ %%

\section{Funds requested (German text, see Appendix)}

\subsection{Staff costs}
The detailed arguments for the staff costs are to be found in the initial proposal. 
For the reasons given there, we need also in future:
\begin{dlist}
   \item A scientific-technical cooperator (Roland Hendel) for 3 years 
   according to TV-L~13
   \item A scientific cooperator (Christoph K\"ostler) for 3 years 
   according to TV-L~13/2
\end{dlist}

\subsection{Scientific instrumentation}
In general we do the calculations on the supercomputers at LRZ (Garching) and HLRS (Stuttgart) and KIT (Karlsruhe). 
However, for development and testing it is highly desirable to run the calculations on less processors, 
but without delay due to the supercomputers' queue system. 
Therefore, we apply for a server solution with 12 processor cores. (If funded, 20 cores will be available at the time of purchase at roughly the same price.)  
%Even for test runs, we need more than 10~GB RAM to get a reasonable grid resolution, so we chose 16~GB.

%\vspace{0.2ex}
\noindent
\begin{tabular*}{\textwidth}{l@{\extracolsep\fill}rrr}
\multicolumn{2}{l}
   {\textbf{Instrument A} (JECOSYS Server Pro 2300-T-1366: 2x~Intel SixCore Xeon, 
   16 GB RAM)} && \\
Offer by JECOSYS Redlich-IT & dated Jan. 2, 2012 & 4427 &\\ \cline{3-4}
& Purchase Price 4.2 & 4427 & \textbf{EUR}\\ \cline{3-4}\cline{3-4}
\end{tabular*}

\subsection{Consumables}
A multitude of colored pictures are necessary for the evaluation of our model computations. 
Therefore the consumables consist mainly of print cartridges and photo-conductors. 
The expected costs amount to about 1700~EUR per year. 
For local archiving of model run data, we also need external disk space of 10~TB per year, 
yielding costs of about 300~EUR per year.

%\vspace{0.3ex}
\noindent
\begin{tabular*}{\textwidth}{l@{\extracolsep\fill}llllllrr}
&&&&&& Printing material  & 5100 &\\ \cline{8-9}
&&&&&& External HDDs & 900 &\\ \cline{8-9}
&&&&&& Total 4.3 & 6000 & \textbf{EUR}\\ \cline{8-9}\cline{8-9}
\end{tabular*}\\

\subsection{Travel}
The personal exchange of ideas during scientific conferences (EGU, AGU, International Symposium on Andean Geodynamics (ISAG), Goldschmidt Conference) is extremely important and irreplaceable for our further work. 
We want to continue the joint work in the group of Terra-developers
which meet once per year in a European country.
As usual, we want to take part in the Fall Meetings of the High performance computing centers in Stuttgart and Garching.\\

% \vspace{0.1ex}
\noindent
\begin{tabular*}{\textwidth}{l@{\extracolsep\fill}lrrr}
\multicolumn{2}{l}
   {\textbf{First year (2013)}} &&& \\
1 & travel to the AGU Fall Meeting (San Francisco) & & &\\ 
  & connected with a working time with John Baumgardner && 3500 &\\ \cline{4-5}
1 & travel to the Goldschmidt Conference (Florence) && 1500 &\\ \cline{4-5}
4 & travels within Germany/Europe (destinations, see above) && 1000 &\\ \cline{4-5}
\multicolumn{2}{l}
   {\textbf{Second year (2014)}} &&& \\
1 & travel to the AGU Fall Meeting (San Francisco) & &\\ 
  & connected with a working time with John Baumgardner && 3500 &\\ \cline{4-5}
1 & travel to the EGU Spring Meeting (Vienna) && 1500 &\\ \cline{4-5}
4 & travels within Germany/Europe (destinations, see above) && 1000 &\\ \cline{4-5}
\multicolumn{2}{l}
   {\textbf{Third year (2015)}} &&& \\
2 & travels to the ISAG && 2000 &\\ \cline{4-5}
1 & travel to the AGU Fall Meeting (San Francisco) && &\\ 
  & connected with a working time with John Baumgardner && 3500 &\\ \cline{4-5}
1 & travel to the EGU Spring Meeting (Vienna) && 1500 &\\ \cline{4-5}
4 & travels within Germany/Europe (destinations, see above) && 1000 &\\ \cline{4-5}
\hfill & & Total 4.4 & 20000 & \textbf{EUR}\\ \cline{4-5}\cline{4-5}
\end{tabular*}\\

\subsection{Publication expenses}
Especially for color figures, the printing expenses are high.\\
\begin{tabular*}{\textwidth}{l@{\extracolsep\fill}llllllrr}
\hfill &&&& Total 4.5 & 3~x~750 EUR &=& 2250 & \textbf{EUR}\\ \cline{8-9}\cline{8-9}
\end{tabular*}\\

%% ------------------------------------------------------------------------ %%
%
%  SECTION 5
%
%% ------------------------------------------------------------------------ %%

\section{Prerequisites for carrying out the project 
   (Voraussetzungen f\"ur die Durchf\"uhrung des Vorhabens) }

\subsection{Our team (Zusammensetzung der Arbeitsgruppe)}
\begin{dlist}
   \item Jonas Kley, Prof. Dr. (Project leader), tasks cf. Section 4.1.
   \item Uwe Walzer, Prof. Dr. (Second applicant), tasks cf. Section 4.1.
   \item Lothar Viereck-G\"otte, Prof. Dr. (Third applicant), tasks cf. Section 4.1.
   \item Markus M\"uller, Dr., MPI Biogeochemie, FE-discretization and local grid refinement in Terra.
   \item Roland Hendel, Dipl.-Math., tasks cf. Section 4.1.
   \item Christoph K\"ostler, Dr., tasks cf. Section 4.1.
\end{dlist}
We ask for funds only for Roland Hendel and Christoph K\"ostler (cf. Section 4.1).

\subsection{Cooperation with other scientists (Zusammenarbeit mit anderen Wissenschaftlern)}
\label{sec:coop}
In \emph{Germany}, we will cooperate with experts on Andean evolution from GFZ Potsdam, building on earlier successful collaboration. 
We are also in contact with other colleagues who work on problems relevant for this proposal.

\vspace{0.5ex}
\noindent
\begin{tabular*}{\textwidth}{l@{\extracolsep\fill}lll}
Oncken, Onno	& Prof. Dr.	& GFZ Potsdam 	& Andean orogeny \\
Sobolev, Stephan& Dr.		& GFZ Potsdam 	& Numerical modeling of orogens\\
Hindle, David	& Dr.		& Univ. Jena & Modeling of orogen-scale strain\\
Davies, Huw*& Prof. Dr.	& Univ. Cardiff & Modeling of mantle convection\\
Davies, Rhodri*& Dr.	& Imp. Coll. London & Particle tracking, non-uniform grid\\
Mohr, Marcus*& Dr.	& Univ. Munich & Analytical benchmarks, MG-implementation \\
Bollada, Peter*& Dr.	& Univ. Leeds & Free-slip boundary conditions\\
Quevedo, Leonardo*& Dr.	& Univ. Sydney & Plate motion reconstruction\\
Bunge, Hans-Peter& Prof. Dr.	& LMU M\"unchen & Mantle convection, High Performance Computing\\
Schmeling, Harro& Prof. Dr.	& Univ. Frankfurt/M. & Modeling of mantle convection, plumes,\\
   &&& Dynamics of subduction and orogeny\\
Hansen, Ulrich& Prof. Dr.	& Univ. M\"unster & Numerical modeling of mantle convection\\
Spohn, Tilman& Prof. Dr.	& DLR Berlin & Planetology, planetary interiors\\
\end{tabular*}

\noindent
In \emph{South America}, we will rely on our long-standing cooperation 
with the Buenos Aires and Salta Universities. Collaborators will include

\vspace{0.5ex}
\noindent
\begin{tabular*}{\textwidth}{l@{\extracolsep\fill}lll}
Rossello, Eduardo A.** & Prof. Dr.	& Univ. Buenos Aires & Regional geology and tectonics, seismology\\
Monaldi, C\'esar Rub\'en** & Dr.		& Univ. Salta & Regional geology, tectonics \\
Salfity, J.~A.**		& Prof. Dr.	& Univ. Salta & Regional geology, stratigraphy and basin evolution\\
Carlotto, V.**		& Dr.		& Instituto Geol. Min.	& Regional geology, stratigraphy, tectonics \\
   &&y Metal. Lima &\\
\end{tabular*}

\noindent
J. Kley and D. Hindle will continue their close cooperation with the five persons denoted by **.

\noindent
U. Walzer cooperates very productively with J.~Baumgardner (San Diego, USA, cf. 8.5).
C.~K\"ostler, R.~Hendel and M.~M\"uller will continue their computational cooperation with the five people denoted by *.

\subsection{Scientific equipment (Apparative Ausstattung)}
Since 2001, the HLRS (Stuttgart) granted us computing time on NEC SX and Cray Opteron machines amounting to 3.8 million Euros. 
Since 2006, we got ca. 1100000 CPU-h at the LRZ (Garching). 
Until 31.07.2014, we have been granted with 550008 CPU-h at Steinbuch Centre for Computing (SCC) in Karlsruhe.

\subsection{Running costs for materials (Laufende Mittel f\"ur Sachausgaben)}
Expenses for office materials, photocopies etc. of about 500 EUR per year 
as well as communication costs and the management of the access machines
will be accounted for by the Institut f\"ur Geowissenschaften, Jena University.

\subsection{Conflicts of interest with commercial activities \\
   (Interessenkonflikte bei wirtschaftlichen Aktivit\"aten)}
There are no such conflicts.

\subsection{Other requirements (Sonstige Voraussetzungen)}
We intend to run the project for another three years. 
Requirements which are fulfilled are mentioned in 5.3.

%% ------------------------------------------------------------------------ %%
%
%  SECTION 6
%
%% ------------------------------------------------------------------------ %%

\section{Declarations (Erkl\"arungen)}
\begin{dlist}
\item We have not requested funding for this project from any other sources. 
   In the event that we submit such a request, we will inform the Deutsche Forschungsgemeinschaft immediately.
\item The DFG liaison officer of the Friedrich-Schiller University of Jena, Prof. Dr. Roland M\"ausbacher, has been informed about the present application.
\end{dlist}

%% ------------------------------------------------------------------------ %%
%
%  SECTION 7
%
%% ------------------------------------------------------------------------ %%

\section{Signatures (Unterschriften)}
{\small
\vspace{2.0cm}
Prof. Dr. Jonas Kley (Project leader) 
\hspace{0.3cm}
Prof. Dr. Uwe Walzer (2nd appl.)
\hspace{0.3cm}
Prof. Dr. Lothar Viereck-G\"otte (3rd appl.)
}
%% ------------------------------------------------------------------------ %%
%
%  SECTION 8
%
%% ------------------------------------------------------------------------ %%

\section{List of Attachments}
\subsection{Curriculum Vitae Jonas Kley}
\subsection{Curriculum Vitae Uwe Walzer}
\subsection{Curriculum Vitae Lothar Viereck-G\"otte}
\subsection{Proposal-related list of publications 1997-2012, Jonas Kley}
\subsection{Proposal-related list of publications 1997-2012, Uwe Walzer}
\subsection{Proposal-related list of publications 2004-2012, Lothar Viereck-G\"otte}
\subsection{Recent publication (K\"ostler et~al.), submitted to Comp. Geodyn.}
\subsection{Recent publication (M\"uller \& K\"ostler), submitted to SIAM}
\subsection{Staff questionnaire Roland Hendel}
\subsection{Staff questionnaire Christoph K\"ostler}
\subsection{Instrumentation quote, Compute-Server}
%\subsection{Ethics statement}

%% ------------------------------------------------------------------------ %%
%
%  APPENDIX
%
%% ------------------------------------------------------------------------ %%

\section*{Appendix: German parts of the proposal}
\input{anden11de}

\selectlanguage{USenglish}

%% ------------------------------------------------------------------------ %%
%
%  REFERENCE LIST AND TEXT CITATIONS
%
%% ------------------------------------------------------------------------ %%

%\bibliographystyle{abbrv}
\bibliographystyle{abbrvshort}
%\bibliographystyle{spphys}
\bibliography{wz,ck}

\end{document}
