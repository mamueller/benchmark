%\documentclass[11pt,twoside]{book}    % 
\documentclass[11pt,a4paper,twoside]{article}    % 
\usepackage{url}
%\usepackage{tex4ht}
\usepackage[round]{natbib}% Literaturzitate mit Autor und Jahr im Text
\usepackage{a4}	% a4 has a higher and a little bit wider textarea
%\usepackage{cite,textcomp,color,ngerman}
\usepackage{amsmath,amssymb}
\usepackage[algo2e,ruled]{algorithm2e} % Algorithmen in Boxen
\usepackage[breaklinks]{hyperref}   % doesn't work but using it after
                                    % typedref/amsthm gives errors
\usepackage{amsthm}
\usepackage[nottoc]{tocbibind}

% mathematical commands
\newcommand{\mb}{\mathbf}
\newcommand{\pdiff}[2]{\frac{\partial#1}{\partial #2}}
\newtheorem{thm}{Theorem}

% space-saving list environment
\newenvironment{dlist}
   {\begin{list}
      {$\bullet$}
      {
      \setlength{\topsep}{0.5ex}
      \setlength{\partopsep}{0.0ex}
      \setlength{\parsep}{0.5ex}
      \setlength{\itemsep}{0.0ex}
      \setlength{\itemindent}{3.0ex}
      \setlength{\leftmargin}{0.0ex}
      \setlength{\labelsep}{1.0ex}
      }
   }
   {\end{list}}


\begin{document}

\subsection{Modeling Convection and Evolution of the Earth's Mantle Using Terra}

One of the earliest three-dimensional numerical model was the 
spherical-shell model Terra, developed by \citet{Baumgardner1983,Baumgardner1985a}.
It uses a finite-element discretization on a triangular grid, and it utilizes
an efficient multigrid algorithm to solve for the velocity. 
It was parallelized by \citet{Bunge1995} through message passing and
domain decomposition in two of three dimensions. A first study of
convection with Earth-like Rayleigh number of $10^8$ and depth-dependent
viscosity was done by \citet{Bunge1997}. At the same time, \citet{Yang1997}
improved the multigrid algorithm with matrix-dependent transfer operators to
represent varying coefficients properly on coarse grids. With this code,
\citet{Richards2001} investigated surface mobility as a function of viscosity
variation and yield stress, and \citet{Reese2005} explored a parameter range of 
$\Delta \eta$ between $10^5$ and $5 \times 10^7$ with an internal Rayleigh
number of $10^6$. \citet{Walzer2004a} derived a viscosity profile, with three
high-viscosity and three low-viscosity zones and steep gradients, based on 
postglacial rebound, mantle mineralogy, seismic tomography, thermodynamics
and high-pressure geophysics. However, some restrictions were put into this
viscosity model because of numerical reasons, especially regarding lateral
variations due to temperature-dependence. Their model derived the evolution of
self-consistent oceanic plates in connection with the thermal evolution of the
spherical shell of the Earth. In this model it was necessary to assume that the
oceanic lithosphere is also a \emph{chemical} boundary layer and that its
existence is not determined by the temperature dependence of viscosity alone.

\citet{Walzer2008a,Walzer2008b} again showed the importance of these viscosity
variations for plate tectonics and surface mobility on Earth and incorporated
chemical differentiation of continents and, as a complement, of the
depleted MORB mantle (DMM). In their model, continents evolve by the interplay
of chemical differentiation and convection/mixing, without the requirement
of modified boundary conditions on the outer surface of the shell. DMM is partly
stirred into the other mantle reservoirs, resulting in a marble-cake mantle with
a high concentration of DMM in the asthenosphere \citep{Walzer2011}.

Terra was also applied by \citet{Oldham2004} to investigate layered convection 
and by \citet{Bunge2005,Davies2005a,Davies2005,Davies2009} to study thermally driven mantle plumes.
To reach a higher grid resolution in the upper mantle, \citet{Davies2008} added
a multi-resolution multigrid method to Terra.

\citet{Phillips2005,Phillips2007,Coltice2007,Phillips2009,Phillips2010} studied the influence of mobile continents on mantle temperture and convective wavelength.

\citet{Bunge2001} used Terra to calculate mantle flow in a circulation model
to further constrain seismic tomographic imaging of the mantle. 
They further did inverse modeling with data assimilation for the past 200~Ma \citet{Bunge2002,Bunge2003} 
to infer mantle flow and structure from seismic tomography and plate motion history.
\citet{Schuberth2009,Schuberth2009a} also investigated thermal and elastic properties and heterogeneities within the mantle to explain velocity models based on global seismic tomography.

\citet{Iaffaldano2007} coupled Terra to a lithospheric model to study plate coupling at the Nazca-South America
convergent margin.

\subsection{Inf-sup in Spherical Shell}
\citet{Tabata2000} extended the proof of unique solvability to 
Ray\-leigh-B\'enard convection in a spherical shell with free-slip boundary
conditions and later also to variable-viscosity convection
\citep{Tabata2002,Tabata2006}. 
As mentioned before, although they use a slightly different formulation 
of the bilinear form $a$, their results should be valid for our formulation
as well. The viscosity in their model is supposed to be a continuously
differentiable function 
of position, time and temperature. It is also confined to an interval 
between extremal values on which the error estimates depend. However, 
the inf-sup constant $\beta$ in their model decreases linearly 
with the overall viscosity contrast.


%The convection code Terra has been 

%% ------------------------------------------------------------------------ %%
%
%  REFERENCE LIST AND TEXT CITATIONS
%
%% ------------------------------------------------------------------------ %%

\bibliographystyle{abbrvnat}
%\bibliographystyle{plainnat}
%\bibliographystyle{spphys}
\bibliography{wz,ck}


\end{document}
